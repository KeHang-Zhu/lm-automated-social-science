
\documentclass[12pt]{article}
\usepackage{geometry}
\usepackage{pgfplots}
\usepackage{tabularx}
\usepackage{array}
\usepackage{multirow}
\usetikzlibrary{arrows,positioning,automata,arrows.meta, shapes.geometric,decorations.pathreplacing, calc}
\usepackage{tikz}
\geometry{top=1in, bottom=1in, left=1in, right=1in}
\setlength{\parindent}{0pt}

\definecolor{headercolor}{RGB}{142, 180, 227} 
\definecolor{rowcolor}{RGB}{230, 240, 255} 

\newcolumntype{L}{>{\raggedright\arraybackslash}p{0.5\textwidth}}
\newcolumntype{R}{>{\raggedleft\arraybackslash}p{0.5\textwidth}}

\begin{document}
\begin{table}[ht]
\centering
\caption{Operationalizations of the variables for the experiment simulated to generate Figure \ref{fig:two people deciding whether or not to watch a movie-scm} for two people deciding whether or not to watch a movie.}
\

\begin{tabularx}{\textwidth}{|c|c|X|}
\hline

\multirow{4}{*}{\parbox{5cm} { \centering `decision to watch the movie' (watch-movie)}} & Variable Type & binary  \\ \cline{2-3}
                        &  Units   &  binary decision   \\ \cline{2-3}
                        & Levels & [`0', `1'] \\ \cline{2-3}
                        & Summary Statistics   & mean: 0.6, std: 0.55, min: 0.0, 25\%: 0.0, 50\%: 1.0, 75\%: 1.0, max: 1.0    \\ \hline
\end{tabularx}\\
                


\begin{tabularx}{\textwidth}{|c|c|X|}
\hline
\multirow{5}{*}{\parbox{5cm} { \centering the number of movies person 2 has seen that day already \\(num-mov-p2)}} & Variable Type & count  \\ \cline{2-3}
                        & Proxy Attribute Name   & number of movies you've watched today   \\ \cline{2-3}
                        & Varied Attribute Levels (5)  & [`0', `2', `4', `6', `7']   \\ \cline{2-3}
                        & Units  & count of movies   \\ \cline{2-3}
                        & Relevant Agent   & person 2   \\ \hline
\end{tabularx}\\
    \begin{minipage}{\textwidth}
    \begin{footnotesize}
      \emph{Notes:} The proxy attribute and one of its values are directly provided to the relevant agent in each simulation.
      The number of simulations is all possible combinations of the varied attribute values for the causal variables.
    \end{footnotesize}
    \end{minipage}
\end{table}
\end{document}
